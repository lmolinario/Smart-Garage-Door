\chapter{Stato dell'Arte}

\section{Panoramica dei sistemi di automazione domestica}

Negli ultimi anni, il settore dell’automazione domestica ha conosciuto una rapida e costante evoluzione grazie alla crescente diffusione delle tecnologie di comunicazione wireless e dei microcontrollori connessi in rete.  
L’avvento dell’\textbf{Internet of Things (IoT)} ha reso possibile l’interconnessione di dispositivi eterogenei, favorendo la nascita di ecosistemi digitali in grado di acquisire, elaborare e condividere dati in tempo reale \cite{Atzori2010, Gubbi2013}.  

I moderni sistemi di \textbf{smart home} si sono progressivamente trasformati da semplici soluzioni di controllo remoto a piattaforme distribuite capaci di apprendere le abitudini dell’utente e adattarsi automaticamente al contesto operativo.  
Questi sistemi sfruttano una combinazione di sensori, attuatori e interfacce digitali per ottimizzare comfort, sicurezza ed efficienza energetica.  
Come osservato da Holler et al. \cite{Holler2014}, la convergenza tra comunicazione macchina–macchina (M2M) e Internet ha posto le basi per l’intelligenza ambientale, aprendo la strada a soluzioni integrate che riducono l’intervento umano nelle operazioni quotidiane.

In tale contesto, le soluzioni dedicate all’automazione di porte, cancelli e garage rappresentano un campo applicativo consolidato ma ancora in espansione.  
L’integrazione di moduli Wi-Fi, servizi cloud e applicazioni mobili ha reso possibile il controllo remoto e la gestione automatizzata degli accessi, ponendo però nuove sfide in termini di interoperabilità, sicurezza e affidabilità.  
Le architetture proposte in letteratura e sul mercato convergono verso modelli distribuiti, in cui la capacità di comunicazione tra nodi e la latenza di risposta costituiscono parametri fondamentali per la qualità complessiva del sistema \cite{Palattella2016, Piyare2013}.

\section{Soluzioni commerciali esistenti}

Le principali soluzioni disponibili sul mercato possono essere ricondotte a due macro-categorie:
\begin{itemize}
    \item \textbf{Sistemi proprietari}, sviluppati da aziende specializzate e basati su infrastrutture chiuse, con comunicazioni centralizzate su server cloud (es. Chamberlain, Tailwind, Nexx);
    \item \textbf{Sistemi aperti o compatibili}, che si integrano con piattaforme standard come Google Home, Alexa o Home Assistant, e adottano protocolli interoperabili quali \textbf{MQTT} o \textbf{Zigbee}.
\end{itemize}

\subsection{Chamberlain \textit{MyQ}}
Il sistema \textit{MyQ} di Chamberlain è una delle soluzioni più diffuse per il controllo remoto delle porte da garage.  
Il dispositivo utilizza un modulo Wi-Fi integrato per connettersi a un’infrastruttura cloud proprietaria, accessibile tramite applicazione mobile.  
L’utente può verificare lo stato della porta, ricevere notifiche e pianificare aperture automatiche.  
Nonostante la buona stabilità operativa, l’architettura chiusa limita l’integrazione con altri sistemi e impedisce il funzionamento in assenza di connessione Internet, generando dipendenza dal cloud e vincoli di privacy.

\begin{figure}[h!]
    \centering
    \includegraphics[width=0.65\textwidth]{images/Chamberlain.png}
    \caption[Chamberlain MyQ – sistema di automazione per porte da garage]
    {Esempio del sistema \textbf{MyQ} di Chamberlain.  
    Fonte: \textit{Chamberlain Group Inc., MyQ Official Product Documentation} (consultato 2025).}
    \label{fig:myq}
\end{figure}

\subsection{Tailwind iQ3}
Tailwind iQ3 adotta un approccio ibrido, combinando la comunicazione cloud con la connessione Bluetooth Low Energy (BLE) del dispositivo mobile.  
Il sistema è in grado di aprire automaticamente il garage quando rileva la presenza dell’auto associata, sfruttando la prossimità BLE.  
Pur garantendo una buona esperienza utente, tale soluzione presenta vincoli di compatibilità con smartphone specifici e un’affidabilità ridotta in ambienti esterni con ostacoli o interferenze elettromagnetiche.


\begin{figure}[h!]
    \centering
    \includegraphics[width=0.65\textwidth]{images/Tailwind_iQ3.png}
    \caption[Tailwind iQ3 – sistema di automazione con BLE]
    {Esempio del sistema \textbf{Tailwind iQ3}.  
    Fonte: \textit{Tailwind Technologies Inc., Tailwind iQ3 Product Documentation} (consultato 2025).}
    \label{fig:tailwind}
\end{figure}

\subsection{Nexx Garage}
Nexx Garage propone un dispositivo Wi-Fi economico che può essere integrato su meccanismi di apertura preesistenti.  
Il sistema consente il controllo remoto, l’automazione di prossimità basata su GPS e il supporto ai comandi vocali.  
Tuttavia, la sua dipendenza da servizi cloud esterni per il monitoraggio e le notifiche genera problemi legati alla sicurezza dei dati, alla continuità di servizio e ai costi di mantenimento a lungo termine.  
In particolare, la gestione remota attraverso infrastrutture centralizzate comporta rischi di latenza e vulnerabilità nel trasferimento di dati sensibili \cite{Palattella2016}.

\begin{figure}[h!]
    \centering
    \includegraphics[width=0.65\textwidth]{images/Nexx_Smart.png}
    \caption[Nexx Garage – controller Wi-Fi per porte da garage]
    {Esempio del sistema \textbf{Nexx Garage}.  
    Fonte: \textit{Nexx Smart Home Inc., Nexx Garage Product Documentation} (consultato 2025).}
    \label{fig:nexx}
\end{figure}

\section{Analisi dei competitors}
Per valutare in modo rigoroso le alternative presenti sul mercato e posizionare correttamente il sistema \textit{Smart Garage Door} rispetto alle soluzioni commerciali esistenti, è stata adottata la metodologia di analisi \textbf{SWOT} (\textit{Strengths, Weaknesses, Opportunities, Threats}).  
Tale metodologia, ampiamente utilizzata nel design dei sistemi IoT e nei processi di technology assessment, consente di analizzare ciascun competitor considerando non solo gli aspetti tecnici (funzionalità, prestazioni, architettura), ma anche quelli strategici quali rischi, opportunità di integrazione, vincoli di adozione e prospettive di miglioramento.

L’obiettivo non è un confronto commerciale, bensì una valutazione tecnica strutturata che permetta di:
\begin{itemize}
    \item identificare i limiti progettuali delle alternative esistenti;
    \item evidenziare aree in cui il progetto proposto apporta un miglioramento concreto;
    \item individuare opportunità e minacce legate a scelte architetturali simili;
    \item giustificare in modo formale le decisioni progettuali adottate nella fase di design.
\end{itemize}

Le tabelle SWOT che seguono sintetizzano l’analisi per i principali competitor identificati nel mercato attuale.


\subsection{Chamberlain \textit{MyQ}}
Il sistema \textit{MyQ} costituisce una delle soluzioni commerciali più diffuse per il controllo remoto delle porte da garage. Opera mediante un'infrastruttura cloud proprietaria e un'applicazione mobile dedicata.
\subsection{Chamberlain \textit{MyQ}}

\begin{table}[h!]
\centering
\caption{Analisi SWOT del sistema Chamberlain MyQ}
\label{tab:swot_myq}
\renewcommand{\arraystretch}{1.4}

\begin{tabular}{|p{0.45\textwidth}|p{0.45\textwidth}|}
\hline
\multicolumn{1}{|c|}{\textbf{Strengths}} & 
\multicolumn{1}{c|}{\textbf{Weaknesses}} \\
\hline
Interfaccia utente curata; notifiche affidabili; prodotto maturo e diffuso. &
Dipendenza completa dal cloud; assenza di funzionamento offline; scarsa interoperabilità; costo elevato. \\
\hline

\multicolumn{1}{|c|}{\textbf{Opportunities}} &
\multicolumn{1}{c|}{\textbf{Threats}} \\
\hline
Integrazione futura con ecosistemi standard; estensione a più modelli di motori. &
Rischi privacy; latenza e failure del cloud; concorrenza di soluzioni open-source più economiche. \\
\hline
\end{tabular}

\end{table}


\subsection{Tailwind iQ3}
Tailwind iQ3 adotta un modello ibrido basato su cloud e Bluetooth Low Energy (BLE), fornendo automazioni di prossimità.

\begin{table}[h!]
\centering
\caption{Analisi SWOT del sistema Tailwind iQ3}
\label{tab:swot_tailwind}
\renewcommand{\arraystretch}{1.4}

\begin{tabular}{|p{0.45\textwidth}|p{0.45\textwidth}|}
\hline
\multicolumn{1}{|c|}{\textbf{Strengths}} & 
\multicolumn{1}{c|}{\textbf{Weaknesses}} \\
\hline
Automazione BLE; installazione semplice; compatibile con Google Home. &
Compatibilità BLE limitata; interferenze frequenti; dipendenza dal cloud. \\
\hline

\multicolumn{1}{|c|}{\textbf{Opportunities}} &
\multicolumn{1}{c|}{\textbf{Threats}} \\
\hline
Possibile apertura a protocolli standard; ampliamento dispositivo. &
Rischi di sicurezza BLE; instabilità outdoor; vulnerabilità del cloud. \\
\hline
\end{tabular}

\end{table}


\subsection{Nexx Garage}
Nexx Garage propone una soluzione Wi-Fi economica compatibile con sistemi di apertura esistenti.

\begin{table}[h!]
\centering
\caption{Analisi SWOT del sistema Nexx Garage}
\label{tab:swot_nexx}
\renewcommand{\arraystretch}{1.4}

\begin{tabular}{|p{0.45\textwidth}|p{0.45\textwidth}|}
\hline
\multicolumn{1}{|c|}{\textbf{Strengths}} & 
\multicolumn{1}{c|}{\textbf{Weaknesses}} \\
\hline
Compatibile con sistemi preesistenti; supporto Alexa/Google; automazioni GPS. &
Dipendenza da cloud; problemi di latenza; vulnerabilità API; scarsa privacy GPS. \\
\hline

\multicolumn{1}{|c|}{\textbf{Opportunities}} &
\multicolumn{1}{c|}{\textbf{Threats}} \\
\hline
Possibile apertura API; margini per migliorare sicurezza. &
Interruzione servizio in caso di failure cloud; concorrenza di soluzioni Wi-Fi open-source. \\
\hline
\end{tabular}

\end{table}

\section*{Sintesi dell'analisi competitors}

L’analisi sistematica dello stato dell’arte ha evidenziato come le principali soluzioni commerciali per l’automazione delle porte da garage presentino alcune limitazioni strutturali ricorrenti. In particolare, tali dispositivi si basano su architetture \textit{cloud–centric}, nelle quali la logica applicativa, l’autenticazione e il processamento degli eventi risiedono quasi interamente su server remoti. Questa impostazione comporta una scarsa autonomia operativa locale e rende l’utente fortemente dipendente dalla disponibilità della rete Internet e dall’infrastruttura del vendor.

Ulteriori criticità emergono dall’impiego di protocolli di comunicazione eterogenei e spesso proprietari (HTTP/HTTPS, BLE, Zigbee), che risultano poco interoperabili tra ecosistemi differenti. Tale frammentazione contrasta con i principi di apertura, generalità e riusabilità che caratterizzano le architetture IoT moderne, come discusso nelle lezioni teoriche del corso. Inoltre, l’assenza di veri meccanismi di \textit{local fallback} impedisce il funzionamento del sistema in condizioni degradate, violando una delle buone pratiche dei sistemi distribuiti, ovvero la capacità dei nodi periferici di mantenere funzionalità minime anche in presenza di connettività intermittente.

Dal punto di vista economico, i prodotti commerciali analizzati presentano costi medi elevati (200–250\,€), dovuti alla presenza di hub proprietari, servizi premium e infrastrutture cloud integrate. Questo aspetto risulta particolarmente significativo in un contesto accademico e sperimentale, dove si privilegiano soluzioni a basso costo, facilmente replicabili e basate su componenti open-source.

Al contrario, il progetto \textit{Smart Garage Door} si caratterizza per un approccio \textbf{locale-first} e per un’architettura pienamente distribuita, nella quale Arduino e ESP8266 collaborano per garantire resistenza ai guasti, autonomia locale e assenza di \textit{single point of failure}. L’impiego di protocolli standard e aperti (HTTP, MQTT) assicura interoperabilità e indipendenza da piattaforme proprietarie, mentre l’integrazione con sensori fisici (PIR, ultrasuoni) e dati GPS consente di realizzare automazioni affidabili e a bassa incidenza di falsi positivi. La natura open-source della soluzione permette inoltre la trasparenza del codice, la possibilità di estendere il sistema e un’elevata sostenibilità economica e didattica.

Per completezza, si riporta una tabella comparativa che sintetizza le principali differenze tra le soluzioni analizzate.

\begin{table}[H]
\centering
\caption{Confronto sintetico tra le principali soluzioni commerciali}
\label{tab:competitors}
\begin{tabularx}{\textwidth}{@{}lXXXX@{}}
\toprule
\textbf{Caratteristica} & \textbf{MyQ} & \textbf{Tailwind iQ3} & \textbf{Nexx Garage} & \textbf{Smart Garage Door (progetto)} \\
\midrule
Architettura & Cloud proprietario & Cloud + BLE & Cloud Wi-Fi & Locale + opzionale cloud \\
Interoperabilità & Bassa & Media & Media & Alta (MQTT/HTTP) \\
Funzionamento offline & No & Limitato & No & Sì (ESP8266 + Arduino) \\
Automazioni ingresso & GPS app & BLE & GPS & GPS + sensori locali \\
Automazioni uscita & No & No & No & PIR + logica locale \\
Privacy & Moderata & Medio-bassa & Bassa & Alta (locale-first) \\
Costo medio & 200–250\,€ & 150–200\,€ & 120–150\,€ & < 150\,€ \\
Open-source & No & No & No & Sì \\
\bottomrule
\end{tabularx}
\end{table}

L’assenza di interoperabilità standard e la dipendenza dai cloud proprietari risultano quindi gli ostacoli principali alla diffusione di soluzioni aperte, resilienti e replicabili. Come evidenziato da Piyare e Lee \cite{Piyare2013}, l’adozione di protocolli leggeri e la decentralizzazione dell’intelligenza locale costituiscono elementi fondamentali per garantire efficienza, affidabilità e riduzione dei consumi nei sistemi IoT a risorse limitate. In questo contesto, il protocollo \textbf{MQTT} \cite{MQTTspec} rappresenta un’alternativa più efficiente rispetto alle architetture REST cloud-based, consentendo comunicazione asincrona, bassissima latenza e robustezza in ambienti domestici.

L’utilizzo della scheda \textbf{ESP8266} \cite{ESP8266}, combinato con la micrologica locale di Arduino, consente di adottare un modello distribuito che massimizza resilienza e continuità del servizio, risultando pienamente coerente con i principi architetturali IoT e con il modello SDLC illustrato nel Capitolo 4.
Questa analisi costituisce la base per le scelte tecnologiche e architetturali descritte nel Capitolo successivo.

\section{Contributo del progetto \textit{Smart Garage Door}}

Alla luce dell’analisi comparativa svolta, il progetto \textbf{Smart Garage Door} si configura come una soluzione innovativa, aperta e pienamente replicabile in ambito accademico, capace di rispondere in modo sistematico alle principali criticità rilevate nelle piattaforme commerciali analizzate.
L’approccio seguito integra principi di progettazione distribuita, attenzione alla sostenibilità economica e adozione di tecnologie standardizzate, con l’obiettivo di massimizzare interoperabilità, affidabilità e trasparenza architetturale.

Le caratteristiche distintive che qualificano il contributo progettuale sono le seguenti:

\begin{itemize}
    \item \textbf{Architettura ibrida locale–remota}, in cui la logica di controllo primaria risiede nei microcontrollori (Arduino Uno ed ESP8266), garantendo operatività anche in assenza di connettività Internet. Tale proprietà, direttamente collegata al requisito \textbf{NFR5}, consente al sistema di evitare dipendenze critiche da server esterni e di mantenere continuità di servizio in scenari degradati.

    \item \textbf{Adozione di componenti open-source e protocolli standard} (\textit{MQTT}, \textit{HTTP/Flask}, \textit{Telegram Bot API}), che favoriscono la piena interoperabilità del sistema e riducono sia i costi sia la complessità di integrazione con futuri moduli o servizi. Tale scelta è coerente con i principi di apertura e riusabilità tipici delle moderne architetture IoT a più livelli.

    \item \textbf{Automazione di prossimità basata su GPS}, implementata tramite il modulo \texttt{FakeGPS} del NodeMCU, preferita al BLE per la maggiore stabilità e accuratezza in ambienti outdoor. Questa soluzione permette di ridurre i falsi positivi in fase di rilevamento ingresso-uscita, in accordo con i principi di efficienza delle risorse e affidabilità funzionale.

    \item \textbf{Interfaccia utente leggera, multi-piattaforma e sicura}, realizzata attraverso le \textbf{Telegram Bot API} \cite{TelegramAPI} e un server \textbf{Flask} \cite{Flask2024}. Ciò consente una gestione intuitiva, multiutente e  facilmente estendibile, in linea con i requisiti di usabilità e di separazione tra livelli architetturali.

    \item \textbf{Budget complessivo inferiore a 150\,€}, che rende il sistema economicamente accessibile, sostenibile e ideale per scenari didattici, sperimentali o di prototipazione rapida. L’utilizzo di componenti quali ESP8266 e Arduino permette inoltre una piena trasparenza della logica hardware e software.
\end{itemize}

Il progetto, nel suo insieme, non si limita a riprodurre funzionalità già presenti negli strumenti commerciali, ma propone un modello \textbf{open-source, scalabile e autonomo}, costruito secondo i criteri di modularità, efficienza e interoperabilità propri dell’ingegneria IoT contemporanea.
L’intero ciclo di sviluppo è stato strutturato secondo il paradigma dello \textbf{System Development Life Cycle (SDLC)} \cite{Pressman2019}, applicato in tutte le sue fasi: analisi dei requisiti, progettazione multilivello, implementazione, test e validazione funzionale.

\vspace{4mm}

\section{Conclusioni della revisione dello stato dell’arte}

La revisione dello stato dell’arte ha mostrato che, nonostante la diffusione di soluzioni commerciali per l’automazione delle porte da garage, tali sistemi rimangono generalmente vincolati a \textbf{architetture chiuse, dipendenti dal cloud} e spesso non interoperabili. Questo limita la possibilità di personalizzare il comportamento del sistema, ne aumenta i costi di mantenimento e riduce la flessibilità nell’integrazione con ecosistemi domotici eterogenei.

In questo contesto, il progetto \textit{Smart Garage Door} introduce un paradigma alternativo, fondato su un'architettura \textbf{decentralizzata, leggera e pienamente controllabile dall’utente finale}. La combinazione di logica locale, protocolli aperti e assenza di dipendenze da infrastrutture proprietarie garantisce maggiore trasparenza, sicurezza e robustezza operativa. La Tabella~\ref{tab:swot} sintetizza tali evidenze attraverso un’analisi SWOT, utile per identificare punti di forza, limiti attuali e potenziali direzioni evolutive.

Questa impostazione si colloca perfettamente nell’alveo della letteratura sui sistemi IoT distribuiti e resilienti \cite{Palattella2016, Piyare2013, Holler2014}, e costituisce una base solida per estensioni future orientate verso ambienti domestici integrati, smart building e scenari multi-dispositivo.

\begin{table}[H]
\centering
\caption{Analisi SWOT del sistema Smart Garage Door}
\label{tab:swot}
\renewcommand{\arraystretch}{1.4}
\begin{tabularx}{\textwidth}{|X|X|}
\hline
\textbf{Strengths} & \textbf{Weaknesses} \\ \hline
Funzionamento locale indipendente dal cloud; uso di componenti open e protocolli standard (ESP8266, MQTT/HTTP, Flask, Telegram); costi hardware ridotti; elevata replicabilità didattica. &
Assenza di gestione utenti avanzata; scalabilità limitata a una singola autorimessa; dipendenza dalla copertura Wi-Fi domestica; sicurezza basata su meccanismi minimi. \\ \hline
\textbf{Opportunities} & \textbf{Threats} \\ \hline
Estensione a multi-porta e multi-utente; integrazione con ecosistemi domotici (Home Assistant, Node-RED); adozione di sensori avanzati; applicabilità a contesti condominiali o industriali. &
Rischi legati alla rete Wi-Fi; possibili interferenze nei sensori PIR/ultrasuoni; dipendenza da servizi esterni (Telegram Bot API); vulnerabilità fisiche dei nodi in ambienti esterni. \\ \hline
\end{tabularx}
\end{table}
