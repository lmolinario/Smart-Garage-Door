\documentclass[11pt,a4paper]{article}

% Packages
\usepackage[utf8]{inputenc}
\usepackage[T1]{fontenc}
\usepackage{lmodern}
\usepackage{geometry}
\usepackage{longtable}
\usepackage{booktabs}
\usepackage{array}
\usepackage{textcomp}
\usepackage{microtype}
\geometry{margin=2cm}

% Title
\title{\textbf{Smart Garage Door \\ Requisiti Funzionali e Non Funzionali}}
\author{Matteo Tuzi \and Lello Molinario}
\date{8 settembre 2025}

\begin{document}

\maketitle

\vspace{-0.5em}
\section*{Scenario e Assunzioni}
Il sistema è concepito per un contesto domestico in cui la porta del garage deve poter essere aperta e chiusa in modo automatico o su richiesta dell’utente. L’utilizzo avviene in abitazioni private, con copertura di rete sufficiente fino all’area del cancello o del box.  

Si assume che il meccanismo di apertura sia compatibile con un controllo digitale tramite contatto elettrico e che sia possibile rilevare lo stato della porta (aperta, chiusa, in movimento). È prevista la presenza di un’interfaccia utente remota per l’interazione e la consultazione dello stato del sistema.  

Nel rispetto dei vincoli del corso, i dispositivi devono essere considerati \textit{battery-powered}, orientando il progetto verso componenti e protocolli di comunicazione a basso consumo.

\vspace{0.5em}
\section*{Obiettivo}
L’obiettivo è realizzare un sistema \textit{IoT} per il controllo intelligente della porta del garage, che migliori comfort, sicurezza e automazione domestica.  
Il sistema deve consentire l’apertura e la chiusura da remoto, l’invio di notifiche automatiche sugli stati, la chiusura temporizzata e la gestione multiutenza.  
Deve garantire affidabilità, basso consumo energetico, costi contenuti e un’architettura modulare, facilmente evolvibile nel tempo.

\vspace{0.5em}
\section*{Requisiti Funzionali}
\renewcommand{\arraystretch}{1.2}
\setlength{\tabcolsep}{5pt}
\begin{longtable}{>{\bfseries}p{0.08\textwidth} p{0.28\textwidth} p{0.30\textwidth} p{0.28\textwidth}}
\toprule
FR & Descrizione & Input & Output \\
\midrule
FR1 & Apertura/chiusura remota della porta & Comando utente remoto & Cambio di stato della porta \\
FR2 & Consultazione dello stato della porta in tempo reale & Richiesta utente & Stato porta (aperta/chiusa/in movimento) \\
FR3 & Notifiche automatiche ai cambi di stato & Evento di variazione & Notifica all’utente \\
FR4 & Chiusura temporizzata automatica & Timer scaduto / nessun movimento rilevato & Comando di chiusura \\
FR5a & Automazione di prossimità in uscita &
Rilevazione dell’intenzione di uscita dall’interno del garage (porta chiusa e movimento verso la soglia) &
Apertura automatica della porta solo se viene confermato un effettivo tentativo di uscita. \\
FR5b & Automazione di prossimità in ingresso &
Rilevazione di un utente autorizzato che si avvicina al garage dall’esterno entro un raggio configurabile &
Apertura automatica della porta subordinata al riconoscimento dell’utente e alla distanza inferiore alla soglia definita. \\
FR6 & Gestione multiutenza & Richiesta di aggiunta o rimozione di utenti & Aggiornamento dell’elenco degli utenti autorizzati \\
FR7 & Comando locale / override & Intervento manuale & Esecuzione immediata del comando \\
FR8 & Rilevazione ostacolo con riapertura & Sensore ostacolo attivo & Arresto e riapertura della porta \\
FR9 & Consultazione log essenziale & Richiesta amministratore & Elenco di eventi e azioni registrate \\
\bottomrule
\end{longtable}

\vspace{0.5em}
\section*{Requisiti Non Funzionali}
\renewcommand{\arraystretch}{1.2}
\begin{tabular}{>{\bfseries}p{0.1\textwidth} p{0.83\textwidth}}
\toprule
NFR1 & Accessibilità continua ai dati, con tempo di conservazione configurabile. \\
NFR2 & Tempo di risposta massimo pari a 1\,s (percentile 95) per comandi e notifiche. \\
NFR3 & Accuratezza superiore al 99\%, con tasso di falsi positivi nella rilevazione di prossimità inferiore all’1\%. \\
NFR4 & Rilevazione della presenza dell’utente entro un raggio massimo di 15\,m. \\
NFR5 & Disponibilità del servizio 24/7 con possibilità di controllo locale in caso di perdita di connessione. \\
NFR6 & Adozione di misure di sicurezza per autenticazione, integrità dei dati e prevenzione degli accessi non autorizzati. \\
NFR7 & Tutela della privacy mediante minimizzazione dei dati trattati e limitazione del periodo di conservazione dei log. \\
NFR8 & Compatibilità con più piattaforme utente e interoperabilità con sistemi di controllo standard. \\
NFR9 & Consumo energetico contenuto, con funzionamento ottimizzato per dispositivi alimentati a batteria. \\
NFR10 & Costo complessivo del sistema non superiore a 150 euro. \\
\bottomrule
\end{tabular}

\vspace{0.5em}
\section*{Tracciabilità FR $\rightarrow$ NFR}
\renewcommand{\arraystretch}{1.2}
\begin{longtable}{>{\bfseries}p{0.1\textwidth} p{0.33\textwidth} p{0.45\textwidth}}
\toprule
FR & Descrizione & NFR impattati \\
\midrule
FR1 & Apertura/chiusura remota & NFR2, NFR6, NFR8, NFR10 \\
FR2 & Stato porta in tempo reale & NFR1, NFR2, NFR3 \\
FR3 & Notifiche e messaggi & NFR1, NFR2, NFR7 \\
FR4 & Auto-close temporizzato & NFR3, NFR5, NFR9 \\
FR5a & Automazione di prossimità in uscita & NFR3, NFR4, NFR9 \\
FR5b & Automazione di prossimità in ingresso & NFR3, NFR4, NFR9 \\
FR6 & Gestione utenti & NFR6, NFR7, NFR8 \\
FR7 & Override locale & NFR5, NFR8 \\
FR8 & Stop su ostacolo & NFR3, NFR5 \\
FR9 & Log eventi & NFR1, NFR6, NFR7 \\
\bottomrule
\end{longtable}

\vspace{0.5em}
\section*{Conclusioni}
Il progetto \textit{Smart Garage Door} integra requisiti funzionali e non funzionali bilanciando semplicità d’uso, sicurezza e sostenibilità.  
La tracciabilità mostra come ogni funzione sia vincolata da criteri qualitativi che ne guidano l’implementazione, assicurando coerenza tra scenario, obiettivi e scelte progettuali.  
Il risultato atteso è un sistema affidabile, economico ed evolvibile, pronto a migliorare comfort e sicurezza in un contesto domestico reale.

\end{document}
